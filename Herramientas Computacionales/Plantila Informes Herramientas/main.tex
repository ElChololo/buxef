\documentclass[12pt,answers]{exam}
\usepackage[spanish]{babel}
\usepackage[utf8]{inputenc}
\usepackage[table,xcdraw]{xcolor}	% Permite escribir en espanol
\usepackage{footnote}
\makesavenoteenv{solution}
\usepackage{amsthm}	
\usepackage{amsmath}
\usepackage{moreenum}
\makeatletter
\renewcommand*\env@matrix[1][*\c@MaxMatrixCols c]{%
  \hskip -\arraycolsep
  \let\@ifnextchar\new@ifnextchar
  \array{#1}}
\makeatother
% Para ecuaciones y demases
\usepackage{amssymb}						% Para ecuaciones y demases
\usepackage{float}							% Para manejar la ubicacion de graficos
\usepackage{verbatim}						% Para escribir codigos
\usepackage{url}								% Para escribir direcciones web
\usepackage{subfig}							% Para poner varias figuras en el mismo marco
\usepackage{psfrag}							% Para hacer reemplazos en las figuras
\usepackage{multicol}
\usepackage{multirow}
\usepackage{bigstrut}
\usepackage{color}
	\definecolor{ceruleanblue}{rgb}{0.16, 0.32, 0.75}
	\definecolor{coolblack}{rgb}{0.0, 0.18, 0.39}
	\definecolor{darkgreen}{rgb}{0.0, 0.2, 0.13}
\usepackage{multirow,hhline}
\usepackage[linkcolor=blue,colorlinks=true]{hyperref}
\def\mathrlap{\mathpalette\mathrlapinternal} 
\def\mathclap{\mathpalette\mathclapinternal}
\def\mathllapinternal#1#2{\llap{$\mathsurround=0pt#1{#2}$}}
\def\mathrlapinternal#1#2{\rlap{$\mathsurround=0pt#1{#2}$}}
\usepackage{graphicx}
\graphicspath{ {images/} }
\usepackage{lipsum}
\usepackage{mdframed}  


%----------coso de los cuadritos
\def\MakeFramed#1{\begin{mdframed}}
\def\endMakeFramed{\end{mdframed}}
\renewcommand{\solutiontitle}{\noindent\textsf{\textbf{Respuesta}}\par\noindent}


%---------------------------diseñito
\pagestyle{headandfoot}					% Opcion para tener headers y footers
\headrule 											% Linea horizontal bajo el header

\firstpageheader{\scriptsize{\includegraphics[scale=0.3]{Pictures/Logo_FCFM.png}}}{} {\scriptsize{Universidad de Chile} \\ \scriptsize{Facultad de Ciencias Físicas y Matemáticas}}
\runningheader{\scriptsize{\includegraphics[scale=0.3]{Pictures/Logo_FCFM.png}}}{\scriptsize Herramientas Computacionales \\ para la Ingeniería y Ciencias\\ \scriptsize{Otoño 2023}} {\scriptsize{Universidad de Chile} \\ \scriptsize{Facultad de Ciencias Físicas y Matemáticas}}

\footrule
\footer{}{\scriptsize{P\'agina \thepage\ de \numpages}}{}
\parindent = 0pt
\renewcommand\partlabel{(\thepartno.)}
\renewcommand\thesubpart{\roman{subpart}}



\begin{document}

\vspace*{\fill}
\begin{center}
    \Huge{ \textbf{Laboratorio 0....}} \\
    \smallskip
    \LaTeX de actividad.....
    \smallskip
\end{center}
\vspace*{\fill}
\begin{flushright}  
    \begin{tabular}{r r}
        \textbf{Alumno}:  &
        \begin{tabular}[t]{r}
            Álvaro Acevedo Díaz 
        \end{tabular}  \\   \\
        \textbf{Rut Alumno}: &
        \begin{tabular}[t]{r}
            20.283.675-5 
        \end{tabular} \\  \\
        \textbf{Curso}:  &
        \begin{tabular}[t]{r}
            Herramientas Computacionales \\ para Ingeniería y Ciencias 
        \end{tabular}  \\   \\
        \textbf{Código Curso}: &  
        \begin{tabular}[t]{r}
            CC1000-5 
        \end{tabular}  \\   \\
          \textbf{Profesor}: &
        \begin{tabular}[t]{r}
            Valentín Muñoz 
        \end{tabular} \\  \\
        \textbf{Profesores Auxiliares}: &
          \begin{tabular}[t]{r}
            Valentina Aravena  \\
            Valentina Montoya 
        \end{tabular} \\  \\
        \textbf{Fecha de Entrega}:  &
        \begin{tabular}[t]{r}
            21 de Abril, 2023 
        \end{tabular} 
    \end{tabular}
\end{flushright}

\newpage
\renewcommand*\contentsname{Tabla de Contenidos}
\tableofcontents
\renewcommand{\listfigurename}{Lista de Figuras}
\listoffigures
\newpage
\section{Introducción}

\section{Resultados}

\section{Uso de la Herramienta}

\section{Síntesis}





\end{document}
